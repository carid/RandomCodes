
% Default to the notebook output style

    


% Inherit from the specified cell style.




    
\documentclass{article}

    
    
    \usepackage{graphicx} % Used to insert images
    \usepackage{adjustbox} % Used to constrain images to a maximum size 
    \usepackage{color} % Allow colors to be defined
    \usepackage{enumerate} % Needed for markdown enumerations to work
    \usepackage{geometry} % Used to adjust the document margins
    \usepackage{amsmath} % Equations
    \usepackage{amssymb} % Equations
    \usepackage{eurosym} % defines \euro
    \usepackage[mathletters]{ucs} % Extended unicode (utf-8) support
    \usepackage[utf8x]{inputenc} % Allow utf-8 characters in the tex document
    \usepackage{fancyvrb} % verbatim replacement that allows latex
    \usepackage{grffile} % extends the file name processing of package graphics 
                         % to support a larger range 
    % The hyperref package gives us a pdf with properly built
    % internal navigation ('pdf bookmarks' for the table of contents,
    % internal cross-reference links, web links for URLs, etc.)
    \usepackage{hyperref}
    \usepackage{longtable} % longtable support required by pandoc >1.10
    \usepackage{booktabs}  % table support for pandoc > 1.12.2
    

    
    
    \definecolor{orange}{cmyk}{0,0.4,0.8,0.2}
    \definecolor{darkorange}{rgb}{.71,0.21,0.01}
    \definecolor{darkgreen}{rgb}{.12,.54,.11}
    \definecolor{myteal}{rgb}{.26, .44, .56}
    \definecolor{gray}{gray}{0.45}
    \definecolor{lightgray}{gray}{.95}
    \definecolor{mediumgray}{gray}{.8}
    \definecolor{inputbackground}{rgb}{.95, .95, .85}
    \definecolor{outputbackground}{rgb}{.95, .95, .95}
    \definecolor{traceback}{rgb}{1, .95, .95}
    % ansi colors
    \definecolor{red}{rgb}{.6,0,0}
    \definecolor{green}{rgb}{0,.65,0}
    \definecolor{brown}{rgb}{0.6,0.6,0}
    \definecolor{blue}{rgb}{0,.145,.698}
    \definecolor{purple}{rgb}{.698,.145,.698}
    \definecolor{cyan}{rgb}{0,.698,.698}
    \definecolor{lightgray}{gray}{0.5}
    
    % bright ansi colors
    \definecolor{darkgray}{gray}{0.25}
    \definecolor{lightred}{rgb}{1.0,0.39,0.28}
    \definecolor{lightgreen}{rgb}{0.48,0.99,0.0}
    \definecolor{lightblue}{rgb}{0.53,0.81,0.92}
    \definecolor{lightpurple}{rgb}{0.87,0.63,0.87}
    \definecolor{lightcyan}{rgb}{0.5,1.0,0.83}
    
    % commands and environments needed by pandoc snippets
    % extracted from the output of `pandoc -s`
    \providecommand{\tightlist}{%
      \setlength{\itemsep}{0pt}\setlength{\parskip}{0pt}}
    \DefineVerbatimEnvironment{Highlighting}{Verbatim}{commandchars=\\\{\}}
    % Add ',fontsize=\small' for more characters per line
    \newenvironment{Shaded}{}{}
    \newcommand{\KeywordTok}[1]{\textcolor[rgb]{0.00,0.44,0.13}{\textbf{{#1}}}}
    \newcommand{\DataTypeTok}[1]{\textcolor[rgb]{0.56,0.13,0.00}{{#1}}}
    \newcommand{\DecValTok}[1]{\textcolor[rgb]{0.25,0.63,0.44}{{#1}}}
    \newcommand{\BaseNTok}[1]{\textcolor[rgb]{0.25,0.63,0.44}{{#1}}}
    \newcommand{\FloatTok}[1]{\textcolor[rgb]{0.25,0.63,0.44}{{#1}}}
    \newcommand{\CharTok}[1]{\textcolor[rgb]{0.25,0.44,0.63}{{#1}}}
    \newcommand{\StringTok}[1]{\textcolor[rgb]{0.25,0.44,0.63}{{#1}}}
    \newcommand{\CommentTok}[1]{\textcolor[rgb]{0.38,0.63,0.69}{\textit{{#1}}}}
    \newcommand{\OtherTok}[1]{\textcolor[rgb]{0.00,0.44,0.13}{{#1}}}
    \newcommand{\AlertTok}[1]{\textcolor[rgb]{1.00,0.00,0.00}{\textbf{{#1}}}}
    \newcommand{\FunctionTok}[1]{\textcolor[rgb]{0.02,0.16,0.49}{{#1}}}
    \newcommand{\RegionMarkerTok}[1]{{#1}}
    \newcommand{\ErrorTok}[1]{\textcolor[rgb]{1.00,0.00,0.00}{\textbf{{#1}}}}
    \newcommand{\NormalTok}[1]{{#1}}
    
    % Define a nice break command that doesn't care if a line doesn't already
    % exist.
    \def\br{\hspace*{\fill} \\* }
    % Math Jax compatability definitions
    \def\gt{>}
    \def\lt{<}
    % Document parameters
    \title{APh190a Final HW}
    
    
    

    % Pygments definitions
    
\makeatletter
\def\PY@reset{\let\PY@it=\relax \let\PY@bf=\relax%
    \let\PY@ul=\relax \let\PY@tc=\relax%
    \let\PY@bc=\relax \let\PY@ff=\relax}
\def\PY@tok#1{\csname PY@tok@#1\endcsname}
\def\PY@toks#1+{\ifx\relax#1\empty\else%
    \PY@tok{#1}\expandafter\PY@toks\fi}
\def\PY@do#1{\PY@bc{\PY@tc{\PY@ul{%
    \PY@it{\PY@bf{\PY@ff{#1}}}}}}}
\def\PY#1#2{\PY@reset\PY@toks#1+\relax+\PY@do{#2}}

\expandafter\def\csname PY@tok@gd\endcsname{\def\PY@tc##1{\textcolor[rgb]{0.63,0.00,0.00}{##1}}}
\expandafter\def\csname PY@tok@gu\endcsname{\let\PY@bf=\textbf\def\PY@tc##1{\textcolor[rgb]{0.50,0.00,0.50}{##1}}}
\expandafter\def\csname PY@tok@gt\endcsname{\def\PY@tc##1{\textcolor[rgb]{0.00,0.27,0.87}{##1}}}
\expandafter\def\csname PY@tok@gs\endcsname{\let\PY@bf=\textbf}
\expandafter\def\csname PY@tok@gr\endcsname{\def\PY@tc##1{\textcolor[rgb]{1.00,0.00,0.00}{##1}}}
\expandafter\def\csname PY@tok@cm\endcsname{\let\PY@it=\textit\def\PY@tc##1{\textcolor[rgb]{0.25,0.50,0.50}{##1}}}
\expandafter\def\csname PY@tok@vg\endcsname{\def\PY@tc##1{\textcolor[rgb]{0.10,0.09,0.49}{##1}}}
\expandafter\def\csname PY@tok@m\endcsname{\def\PY@tc##1{\textcolor[rgb]{0.40,0.40,0.40}{##1}}}
\expandafter\def\csname PY@tok@mh\endcsname{\def\PY@tc##1{\textcolor[rgb]{0.40,0.40,0.40}{##1}}}
\expandafter\def\csname PY@tok@go\endcsname{\def\PY@tc##1{\textcolor[rgb]{0.53,0.53,0.53}{##1}}}
\expandafter\def\csname PY@tok@ge\endcsname{\let\PY@it=\textit}
\expandafter\def\csname PY@tok@vc\endcsname{\def\PY@tc##1{\textcolor[rgb]{0.10,0.09,0.49}{##1}}}
\expandafter\def\csname PY@tok@il\endcsname{\def\PY@tc##1{\textcolor[rgb]{0.40,0.40,0.40}{##1}}}
\expandafter\def\csname PY@tok@cs\endcsname{\let\PY@it=\textit\def\PY@tc##1{\textcolor[rgb]{0.25,0.50,0.50}{##1}}}
\expandafter\def\csname PY@tok@cp\endcsname{\def\PY@tc##1{\textcolor[rgb]{0.74,0.48,0.00}{##1}}}
\expandafter\def\csname PY@tok@gi\endcsname{\def\PY@tc##1{\textcolor[rgb]{0.00,0.63,0.00}{##1}}}
\expandafter\def\csname PY@tok@gh\endcsname{\let\PY@bf=\textbf\def\PY@tc##1{\textcolor[rgb]{0.00,0.00,0.50}{##1}}}
\expandafter\def\csname PY@tok@ni\endcsname{\let\PY@bf=\textbf\def\PY@tc##1{\textcolor[rgb]{0.60,0.60,0.60}{##1}}}
\expandafter\def\csname PY@tok@nl\endcsname{\def\PY@tc##1{\textcolor[rgb]{0.63,0.63,0.00}{##1}}}
\expandafter\def\csname PY@tok@nn\endcsname{\let\PY@bf=\textbf\def\PY@tc##1{\textcolor[rgb]{0.00,0.00,1.00}{##1}}}
\expandafter\def\csname PY@tok@no\endcsname{\def\PY@tc##1{\textcolor[rgb]{0.53,0.00,0.00}{##1}}}
\expandafter\def\csname PY@tok@na\endcsname{\def\PY@tc##1{\textcolor[rgb]{0.49,0.56,0.16}{##1}}}
\expandafter\def\csname PY@tok@nb\endcsname{\def\PY@tc##1{\textcolor[rgb]{0.00,0.50,0.00}{##1}}}
\expandafter\def\csname PY@tok@nc\endcsname{\let\PY@bf=\textbf\def\PY@tc##1{\textcolor[rgb]{0.00,0.00,1.00}{##1}}}
\expandafter\def\csname PY@tok@nd\endcsname{\def\PY@tc##1{\textcolor[rgb]{0.67,0.13,1.00}{##1}}}
\expandafter\def\csname PY@tok@ne\endcsname{\let\PY@bf=\textbf\def\PY@tc##1{\textcolor[rgb]{0.82,0.25,0.23}{##1}}}
\expandafter\def\csname PY@tok@nf\endcsname{\def\PY@tc##1{\textcolor[rgb]{0.00,0.00,1.00}{##1}}}
\expandafter\def\csname PY@tok@si\endcsname{\let\PY@bf=\textbf\def\PY@tc##1{\textcolor[rgb]{0.73,0.40,0.53}{##1}}}
\expandafter\def\csname PY@tok@s2\endcsname{\def\PY@tc##1{\textcolor[rgb]{0.73,0.13,0.13}{##1}}}
\expandafter\def\csname PY@tok@vi\endcsname{\def\PY@tc##1{\textcolor[rgb]{0.10,0.09,0.49}{##1}}}
\expandafter\def\csname PY@tok@nt\endcsname{\let\PY@bf=\textbf\def\PY@tc##1{\textcolor[rgb]{0.00,0.50,0.00}{##1}}}
\expandafter\def\csname PY@tok@nv\endcsname{\def\PY@tc##1{\textcolor[rgb]{0.10,0.09,0.49}{##1}}}
\expandafter\def\csname PY@tok@s1\endcsname{\def\PY@tc##1{\textcolor[rgb]{0.73,0.13,0.13}{##1}}}
\expandafter\def\csname PY@tok@kd\endcsname{\let\PY@bf=\textbf\def\PY@tc##1{\textcolor[rgb]{0.00,0.50,0.00}{##1}}}
\expandafter\def\csname PY@tok@sh\endcsname{\def\PY@tc##1{\textcolor[rgb]{0.73,0.13,0.13}{##1}}}
\expandafter\def\csname PY@tok@sc\endcsname{\def\PY@tc##1{\textcolor[rgb]{0.73,0.13,0.13}{##1}}}
\expandafter\def\csname PY@tok@sx\endcsname{\def\PY@tc##1{\textcolor[rgb]{0.00,0.50,0.00}{##1}}}
\expandafter\def\csname PY@tok@bp\endcsname{\def\PY@tc##1{\textcolor[rgb]{0.00,0.50,0.00}{##1}}}
\expandafter\def\csname PY@tok@c1\endcsname{\let\PY@it=\textit\def\PY@tc##1{\textcolor[rgb]{0.25,0.50,0.50}{##1}}}
\expandafter\def\csname PY@tok@kc\endcsname{\let\PY@bf=\textbf\def\PY@tc##1{\textcolor[rgb]{0.00,0.50,0.00}{##1}}}
\expandafter\def\csname PY@tok@c\endcsname{\let\PY@it=\textit\def\PY@tc##1{\textcolor[rgb]{0.25,0.50,0.50}{##1}}}
\expandafter\def\csname PY@tok@mf\endcsname{\def\PY@tc##1{\textcolor[rgb]{0.40,0.40,0.40}{##1}}}
\expandafter\def\csname PY@tok@err\endcsname{\def\PY@bc##1{\setlength{\fboxsep}{0pt}\fcolorbox[rgb]{1.00,0.00,0.00}{1,1,1}{\strut ##1}}}
\expandafter\def\csname PY@tok@mb\endcsname{\def\PY@tc##1{\textcolor[rgb]{0.40,0.40,0.40}{##1}}}
\expandafter\def\csname PY@tok@ss\endcsname{\def\PY@tc##1{\textcolor[rgb]{0.10,0.09,0.49}{##1}}}
\expandafter\def\csname PY@tok@sr\endcsname{\def\PY@tc##1{\textcolor[rgb]{0.73,0.40,0.53}{##1}}}
\expandafter\def\csname PY@tok@mo\endcsname{\def\PY@tc##1{\textcolor[rgb]{0.40,0.40,0.40}{##1}}}
\expandafter\def\csname PY@tok@kn\endcsname{\let\PY@bf=\textbf\def\PY@tc##1{\textcolor[rgb]{0.00,0.50,0.00}{##1}}}
\expandafter\def\csname PY@tok@mi\endcsname{\def\PY@tc##1{\textcolor[rgb]{0.40,0.40,0.40}{##1}}}
\expandafter\def\csname PY@tok@gp\endcsname{\let\PY@bf=\textbf\def\PY@tc##1{\textcolor[rgb]{0.00,0.00,0.50}{##1}}}
\expandafter\def\csname PY@tok@o\endcsname{\def\PY@tc##1{\textcolor[rgb]{0.40,0.40,0.40}{##1}}}
\expandafter\def\csname PY@tok@kr\endcsname{\let\PY@bf=\textbf\def\PY@tc##1{\textcolor[rgb]{0.00,0.50,0.00}{##1}}}
\expandafter\def\csname PY@tok@s\endcsname{\def\PY@tc##1{\textcolor[rgb]{0.73,0.13,0.13}{##1}}}
\expandafter\def\csname PY@tok@kp\endcsname{\def\PY@tc##1{\textcolor[rgb]{0.00,0.50,0.00}{##1}}}
\expandafter\def\csname PY@tok@w\endcsname{\def\PY@tc##1{\textcolor[rgb]{0.73,0.73,0.73}{##1}}}
\expandafter\def\csname PY@tok@kt\endcsname{\def\PY@tc##1{\textcolor[rgb]{0.69,0.00,0.25}{##1}}}
\expandafter\def\csname PY@tok@ow\endcsname{\let\PY@bf=\textbf\def\PY@tc##1{\textcolor[rgb]{0.67,0.13,1.00}{##1}}}
\expandafter\def\csname PY@tok@sb\endcsname{\def\PY@tc##1{\textcolor[rgb]{0.73,0.13,0.13}{##1}}}
\expandafter\def\csname PY@tok@k\endcsname{\let\PY@bf=\textbf\def\PY@tc##1{\textcolor[rgb]{0.00,0.50,0.00}{##1}}}
\expandafter\def\csname PY@tok@se\endcsname{\let\PY@bf=\textbf\def\PY@tc##1{\textcolor[rgb]{0.73,0.40,0.13}{##1}}}
\expandafter\def\csname PY@tok@sd\endcsname{\let\PY@it=\textit\def\PY@tc##1{\textcolor[rgb]{0.73,0.13,0.13}{##1}}}

\def\PYZbs{\char`\\}
\def\PYZus{\char`\_}
\def\PYZob{\char`\{}
\def\PYZcb{\char`\}}
\def\PYZca{\char`\^}
\def\PYZam{\char`\&}
\def\PYZlt{\char`\<}
\def\PYZgt{\char`\>}
\def\PYZsh{\char`\#}
\def\PYZpc{\char`\%}
\def\PYZdl{\char`\$}
\def\PYZhy{\char`\-}
\def\PYZsq{\char`\'}
\def\PYZdq{\char`\"}
\def\PYZti{\char`\~}
% for compatibility with earlier versions
\def\PYZat{@}
\def\PYZlb{[}
\def\PYZrb{]}
\makeatother


    % Exact colors from NB
    \definecolor{incolor}{rgb}{0.0, 0.0, 0.5}
    \definecolor{outcolor}{rgb}{0.545, 0.0, 0.0}



    
    % Prevent overflowing lines due to hard-to-break entities
    \sloppy 
    % Setup hyperref package
    \hypersetup{
      breaklinks=true,  % so long urls are correctly broken across lines
      colorlinks=true,
      urlcolor=blue,
      linkcolor=darkorange,
      citecolor=darkgreen,
      }
    % Slightly bigger margins than the latex defaults
    
    \geometry{verbose,tmargin=1in,bmargin=1in,lmargin=1in,rmargin=1in}
    
    

    \begin{document}
    
    \author{Zeren Lin}
    \maketitle
    
    

    
    \section{ Definition and
Explanation}\label{definition-and-explanation}

\begin{enumerate}
\def\labelenumi{\arabic{enumi}.}
\item
Rotating wave approximation: terms in the Hamiltonian which oscillate at
high frequencies are neglected. In that sense, the sin() function of the
electric field is approximated by a rotating one. That's why it's called
rotating wave approximation.

Dipole approximation: assume that the electric field is a function of
time only without spatial dependence. Higher orders of the eletric
field-atom interactions, like the gradient or the curvation of the
field, or the recoil motion of the atom, are neglected.


\item

Now since the electric field is quantized, the Rabi frequency depends
not on the classical field strength (amplitude), rather than the number
of photons in the mode. Moreover, due to quantization, the volume of the
system, which determines the miminum available electric field per
photon, comes into play. Also, because of zero-point energy which is
truly a quantum effect, the atom will still exhibit oscillation when
there is no electric field.

As for the eigenvectors, now the atom becomes dressed, which means the
eigenvectors are a combined eigenvectors of the atom and the electric
field.

\item

Rate equation approximation: the polartization dephasing time is much
shorter than the variation of electric field amplitude. Which means one
can assume a constant eletric field amplitude at a certain time, solve
the corresponding susceptibility and then go ahead to another time.
Since the dephasing of polarization is assumed as quickly as it's
generated, it's called rate equation approximation.

\item

\begin{enumerate}
\def\labelenumi{\arabic{enumi}.}
\item
  The population of the two levels (diagonal terms in the density
  matrix), as well as dipole terms (off-digaonal elements in the density
  matrix), will show decay behavior due to interactions with the bath.
  Especially, one of the two decay terms has population difference
  dependence and the other only depends on the population of the upper
  level. The first is the stimulated term and the second accounts for
  spontaneous decay.
\item
  The two-level system will thermalize to the same temperature as the
  bath. More explicitly, due to the finite transition rate, the system
  will show linewidth corresponding to thermal broadening.
\end{enumerate}

\item

On the one hand, becasue of coupling to the bath or some other kind of
interactions, the population difference of the two levels will show a
decay behavior. On the other hand, the gain of the system depends on the
population difference. Therefore, the steady state solution of the
population inversion, thus the gain, will show a saturation behavior.

In the operation of a laser, this means that one can't increase the gain
to infinity by increasing the laser input power, there will be
saturation at some point.
\end{enumerate}

    \section{ Brillouin Amplification}\label{brillouin-amplification}
\begin{enumerate}
\def\labelenumi{\arabic{enumi}.}
  \item Heisenberg equations of motion

$\frac{d b^+}{dt} = \frac{i}{\hbar} [H, b^+] = i \Omega [b^+b, b^+] + i \kappa a A^* e^{i \omega_p t}[b , b^+] = i \Omega b^+ + i \kappa a A^* e^{i \omega_p t}$
(phonon equation)

$\frac{d a}{dt} = \frac{i}{\hbar} [H, a] = i \omega_{s} [a^+a, a] + i \kappa b^+ A e^{-i \omega_p t}[a^+ , a] = -i \omega_{s} a - i \kappa b^+ A e^{-i \omega_p t}$
(Stokes equation)

Add phenomenlogical damping terms

$\frac{d b^+}{dt} =  (i\Omega - \Gamma)b^+ + i \kappa a A^* e^{i \omega_p t}$
(phonon equation)

$\frac{d a}{dt} = (-i\omega_{s} - \gamma) a - i \kappa b^+ A e^{-i \omega_p t}$
(Stokes equation)



 \item Since $\Gamma \gg \gamma$, we can solve the phonon equation assuming
$a$ is a constant first

$b^+ (t) = \int_{-\infty}^t e^{(i\Omega - \Gamma)(t-z)}i \kappa a(z) A^* e^{i \omega_p z} dz \simeq i \kappa  A^* a(t) \int_{-\infty}^t e^{(i\Omega - \Gamma)(t-z)} e^{i \omega_p z} dz$

Plug into the Stokes equation

$\frac{d a}{dt} = (-i\omega_{s} - \gamma) a + \kappa^2  |A|^2 a\int_{-\infty}^t e^{(i\Omega - \Gamma)(t-z)} e^{-i \omega_p (t-z)} dz = (-i \omega_{s} - \gamma) a +  \frac{\kappa^2  |A|^2 }{i(\omega_p-\Omega)+\Gamma} a = \{i (-\omega_{s} - \frac{\kappa^2  |A|^2(\omega_p-\Omega) }{(\omega_p-\Omega)^2+\Gamma^2}) +  (\frac{\kappa^2  |A|^2 \Gamma}{(\omega_p-\Omega)^2+\Gamma^2}-\gamma )\}a$

The optical gain of the Stokes field is given by

$g = \frac{\kappa^2  |A|^2 \Gamma}{(\omega_p-\Omega)^2+\Gamma^2}-\gamma$

The gain has its maximum when

$\omega_p  = \Omega$

Or the energy per photon of the pump field is equal to the energy of the
phonon

  \item The maximum gain is

$g_m = \frac{\kappa^2  |A|^2 }{\Gamma}-\gamma$

Therefore the threshold pumping level is

$|A|^2_{thres}= \frac{\gamma \Gamma}{\kappa^2}$

\end{enumerate}

    \section{Purcell Effect}\label{purcell-effect}
\begin{enumerate}
\def\labelenumi{\arabic{enumi}.}
  \item The Hamiltonian is

$H=E_1 n_1 + E_2 n_2 + \hbar \omega a^+a - \frac{\hbar \Omega_0}{2}  (\sigma_+ a + \sigma_- a^+)$

Then the Heisenberg equation of motion for $n_2$ is

$\frac{d n_2}{dt} = \frac{i}{\hbar} [H, n_2] = -i \frac{\Omega_0}{2}[\sigma_+  , n_2]a-i \frac{\Omega_0}{2}[\sigma_- , n_2]a^+ = i \frac{\Omega_0}{2}\sigma_+ a -i \frac{\Omega_0}{2} \sigma_- a^+$

where

$\frac{d \sigma_- a^+}{dt} = \frac{i}{\hbar} [H, \sigma_- a^+] =\frac{i}{\hbar}E_1[n_1, \sigma_- ]a^++\frac{i}{\hbar} E_2[n_2, \sigma_- ]a^+ +i\omega \sigma_-[a^+a,  a^+]-i \frac{\Omega_0}{2}[\sigma_+ a ,\sigma_- a^+] =\frac{i}{\hbar}E_1 \sigma_- a^+ - \frac{i}{\hbar} E_2\sigma_-a^+ +i\omega \sigma_- a^+-i \frac{\Omega_0}{2}\{(n_2-n_1)a^+ a + n_2\} = -i(\Omega-\omega)\sigma_- a^+  -i \frac{\Omega_0}{2}\{(n_2-n_1)a^+ a + n_2\}$

For anti-nodes of the resonator

$\langle a^+ a\rangle = n_p$

which is hardly affected by the interaction with the atom

Also

$n_1+n_2 =1$ (identity matrix)

Therefore

$\frac{d \sigma_- a^+}{dt} = -i(\Omega-\omega)\sigma_- a^+  -i \frac{\Omega_0}{2} \{(2n_2-1)n_p + n_2\}$

Add damping

$\frac{d \sigma_- a^+}{dt} = (-i(\Omega-\omega)-\frac{1}{2\tau_p})\sigma_- a^+  -i \frac{\Omega_0}{2} \{(2n_2-1)n_p + n_2\}$

Since $\tau_p$ is the fastest rate, we can solve $\sigma_- a^+$ first

$\sigma_- a^+ = -i \frac{\Omega_0}{2} \{(2n_2-1)n_p + n_2\} \int_{-\infty}^t e^{(-i(\Omega-\omega)-\frac{1}{2\tau_p})(t-z)}dz =-i \frac{\Omega_0}{2} \frac{(2n_2-1)n_p + n_2}{i(\Omega-\omega)+\frac{1}{2\tau_p}} $

Therefore

$\frac{d n_2}{dt} = - \frac{\Omega_0^2}{4}\frac{(2n_2-1)n_p + n_2}{-i(\Omega-\omega)+\frac{1}{2\tau_p}}- \frac{\Omega_0^2}{4} \frac{(2n_2-1)n_p + n_2}{i(\Omega-\omega)+\frac{1}{2\tau_p}} = - \frac{\Omega_0^2}{4 \tau_p}\frac{(2n_2-1)n_p + n_2}{(\Omega-\omega)^2+(\frac{1}{2\tau_p})^2} =  \frac{\Omega_0^2}{4 \tau_p}\frac{2n_2-1}{(\Omega-\omega)^2+(\frac{1}{2\tau_p})^2} n_p +\frac{\Omega_0^2}{4 \tau_p}\frac{n_2}{(\Omega-\omega)^2+(\frac{1}{2\tau_p})^2}$

  \item The decay rate of the electron is

$\gamma_e = \frac{\Omega_0^2}{4 \tau_p}\frac{2n_2-1}{(\Omega-\omega)^2+(\frac{1}{2\tau_p})^2} n_p +\frac{\Omega_0^2}{4 \tau_p}\frac{n_2}{(\Omega-\omega)^2+(\frac{1}{2\tau_p})^2}= \frac{\Omega_0^2}{4 }\frac{\frac{\omega}{Q} (2n_2-1)}{(\Omega-\omega)^2+(\frac{\omega}{2Q})^2} n_p + \frac{\Omega_0^2}{4 }\frac{\frac{\omega}{Q}}{(\Omega-\omega)^2+(\frac{\omega}{2Q})^2} $

where the first term in the stimulated term and the second one is the
spontaneous decay

$\gamma_{e, sp} =  \frac{\Omega_0^2}{4 }\frac{\frac{\omega}{Q}}{(\Omega-\omega)^2+(\frac{\omega}{2Q})^2}$

For small detuning compared to optical decay rate

$\gamma_{e, sp} \simeq   \frac{\Omega_0^2Q}{\omega} \propto \frac{Q}{V_{mode}}$

   \item For two-level system in vacuum, the corresponding spontaneous decay
rate is exactly

$\gamma_{e, sp}^{(0)} =   \frac{\Omega_0^2Q}{\omega} $

Then

$\gamma_e = \frac{\Omega_0^2}{4 }\frac{\frac{\omega}{Q} (2n_2-1)}{(\Omega-\omega)^2+(\frac{\omega}{2Q})^2} n_p + \frac{\Omega_0^2}{4 }\frac{\frac{\omega}{Q}}{(\Omega-\omega)^2+(\frac{\omega}{2Q})^2}= \gamma_{e, sp}^{(0)}\frac{(\frac{\omega}{2Q})^2 (2n_2-1)}{(\Omega-\omega)^2+(\frac{\omega}{2Q})^2} n_p + \gamma_{e, sp}^{(0)}\frac{(\frac{\omega}{2Q})^2}{(\Omega-\omega)^2+(\frac{\omega}{2Q})^2} $

For small detuning compared to optical decay rate

$\gamma_e \simeq \gamma_{e, sp}^{(0)}(2n_2-1) n_p + \gamma_{e, sp}^{(0)} \propto \frac{Q}{V_{mode}} \{ (2n_2-1) n_p+1 \}$

The decay rate can be greatly enhanced if $\frac{Q}{V_{mode}}$ is very
large

    \item We can then write the Heisenberg equation of motion for $a^+a$

$\frac{d a^+a}{dt} = \frac{i}{\hbar} [H, a^+a] =-i \frac{\Omega_0}{2} \sigma_+ [a ,a^+a]-i \frac{\Omega_0}{2}\sigma_-[ a^+ ,a^+a] = -i \frac{\Omega_0}{2}\sigma_+ a + i \frac{\Omega_0}{2}\sigma_- a^+ = -\frac{d n_2}{dt}=  \frac{\Omega_0^2}{4 \tau_p}\frac{2n_2-1}{(\Omega-\omega)^2  +(\frac{1}{2\tau_p})^2} a^+a+\frac{\Omega_0^2}{4 \tau_p}\frac{ n_2}{(\Omega-\omega)^2+(\frac{1}{2\tau_p})^2}$

Add damping term

$\frac{d a^+a}{dt} =(  \frac{\Omega_0^2}{4 \tau_p}\frac{2n_2-1}{(\Omega-\omega)^2  +(\frac{1}{2\tau_p})^2} -\frac{1}{\tau})a^+a+\frac{\Omega_0^2}{4 \tau_p}\frac{ n_2}{(\Omega-\omega)^2+(\frac{1}{2\tau_p})^2}$

Since the damping of $\sigma_- a^+$ comes from the damping of the
electron and the damping of the photon, then

$\frac{1}{\tau_p} \simeq \frac{1}{\tau} > \frac{\Omega_0^2}{4 \tau_p}\frac{2n_2-1}{(\Omega-\omega)^2  +(\frac{1}{2\tau_p})^2}$

Therefore

$a^+ a(t)\simeq a^+a(t=-\infty)e^{( \frac{\Omega_0^2}{4 \tau_p}\frac{2n_2-1}{(\Omega-\omega)^2  +(\frac{1}{2\tau_p})^2} -\frac{1}{\tau})t}+\frac{\Omega_0^2}{4 \tau_p}\frac{ n_2(t)}{(\Omega-\omega)^2+(\frac{1}{2\tau_p})^2}\int_{-\infty}^t e^{( \frac{\Omega_0^2}{4 \tau_p}\frac{2n_2-1}{(\Omega-\omega)^2  +(\frac{1}{2\tau_p})^2} -\frac{1}{\tau})(t-z)} dz \simeq a^+a(t=-\infty)e^{ -\frac{1}{\tau}t}+\frac{\Omega_0^2}{4 \tau_p}\frac{ n_2(t)}{(\Omega-\omega)^2+(\frac{1}{2\tau_p})^2} \int_{-\infty}^t e^{  -\frac{1}{\tau}(t-z)} dz \simeq \frac{\Omega_0^2}{4 \tau_p}\frac{ n_2 (t)}{(\Omega-\omega)^2+(\frac{1}{2\tau_p})^2}$

Since $a^+a$ has the same time dependence as $n_2$, if we assume the
optical decay rate is faster than the decay rate of the electron, the
approximation still holds for the decay rate of the photon number in
this mode.

\end{enumerate}


    % Add a bibliography block to the postdoc
    
    
    
    \end{document}
