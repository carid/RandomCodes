
% Default to the notebook output style

    


% Inherit from the specified cell style.




    
\documentclass{article}

    
    
    \usepackage{graphicx} % Used to insert images
    \usepackage{adjustbox} % Used to constrain images to a maximum size 
    \usepackage{color} % Allow colors to be defined
    \usepackage{enumerate} % Needed for markdown enumerations to work
    \usepackage{geometry} % Used to adjust the document margins
    \usepackage{amsmath} % Equations
    \usepackage{amssymb} % Equations
    \usepackage{eurosym} % defines \euro
    \usepackage[mathletters]{ucs} % Extended unicode (utf-8) support
    \usepackage[utf8x]{inputenc} % Allow utf-8 characters in the tex document
    \usepackage{fancyvrb} % verbatim replacement that allows latex
    \usepackage{grffile} % extends the file name processing of package graphics 
                         % to support a larger range 
    % The hyperref package gives us a pdf with properly built
    % internal navigation ('pdf bookmarks' for the table of contents,
    % internal cross-reference links, web links for URLs, etc.)
    \usepackage{hyperref}
    \usepackage{longtable} % longtable support required by pandoc >1.10
    \usepackage{booktabs}  % table support for pandoc > 1.12.2
    

    
    
    \definecolor{orange}{cmyk}{0,0.4,0.8,0.2}
    \definecolor{darkorange}{rgb}{.71,0.21,0.01}
    \definecolor{darkgreen}{rgb}{.12,.54,.11}
    \definecolor{myteal}{rgb}{.26, .44, .56}
    \definecolor{gray}{gray}{0.45}
    \definecolor{lightgray}{gray}{.95}
    \definecolor{mediumgray}{gray}{.8}
    \definecolor{inputbackground}{rgb}{.95, .95, .85}
    \definecolor{outputbackground}{rgb}{.95, .95, .95}
    \definecolor{traceback}{rgb}{1, .95, .95}
    % ansi colors
    \definecolor{red}{rgb}{.6,0,0}
    \definecolor{green}{rgb}{0,.65,0}
    \definecolor{brown}{rgb}{0.6,0.6,0}
    \definecolor{blue}{rgb}{0,.145,.698}
    \definecolor{purple}{rgb}{.698,.145,.698}
    \definecolor{cyan}{rgb}{0,.698,.698}
    \definecolor{lightgray}{gray}{0.5}
    
    % bright ansi colors
    \definecolor{darkgray}{gray}{0.25}
    \definecolor{lightred}{rgb}{1.0,0.39,0.28}
    \definecolor{lightgreen}{rgb}{0.48,0.99,0.0}
    \definecolor{lightblue}{rgb}{0.53,0.81,0.92}
    \definecolor{lightpurple}{rgb}{0.87,0.63,0.87}
    \definecolor{lightcyan}{rgb}{0.5,1.0,0.83}
    
    % commands and environments needed by pandoc snippets
    % extracted from the output of `pandoc -s`
    \providecommand{\tightlist}{%
      \setlength{\itemsep}{0pt}\setlength{\parskip}{0pt}}
    \DefineVerbatimEnvironment{Highlighting}{Verbatim}{commandchars=\\\{\}}
    % Add ',fontsize=\small' for more characters per line
    \newenvironment{Shaded}{}{}
    \newcommand{\KeywordTok}[1]{\textcolor[rgb]{0.00,0.44,0.13}{\textbf{{#1}}}}
    \newcommand{\DataTypeTok}[1]{\textcolor[rgb]{0.56,0.13,0.00}{{#1}}}
    \newcommand{\DecValTok}[1]{\textcolor[rgb]{0.25,0.63,0.44}{{#1}}}
    \newcommand{\BaseNTok}[1]{\textcolor[rgb]{0.25,0.63,0.44}{{#1}}}
    \newcommand{\FloatTok}[1]{\textcolor[rgb]{0.25,0.63,0.44}{{#1}}}
    \newcommand{\CharTok}[1]{\textcolor[rgb]{0.25,0.44,0.63}{{#1}}}
    \newcommand{\StringTok}[1]{\textcolor[rgb]{0.25,0.44,0.63}{{#1}}}
    \newcommand{\CommentTok}[1]{\textcolor[rgb]{0.38,0.63,0.69}{\textit{{#1}}}}
    \newcommand{\OtherTok}[1]{\textcolor[rgb]{0.00,0.44,0.13}{{#1}}}
    \newcommand{\AlertTok}[1]{\textcolor[rgb]{1.00,0.00,0.00}{\textbf{{#1}}}}
    \newcommand{\FunctionTok}[1]{\textcolor[rgb]{0.02,0.16,0.49}{{#1}}}
    \newcommand{\RegionMarkerTok}[1]{{#1}}
    \newcommand{\ErrorTok}[1]{\textcolor[rgb]{1.00,0.00,0.00}{\textbf{{#1}}}}
    \newcommand{\NormalTok}[1]{{#1}}
    
    % Define a nice break command that doesn't care if a line doesn't already
    % exist.
    \def\br{\hspace*{\fill} \\* }
    % Math Jax compatability definitions
    \def\gt{>}
    \def\lt{<}
    % Document parameters
    \title{PS9}
    
    
    

    % Pygments definitions
    
\makeatletter
\def\PY@reset{\let\PY@it=\relax \let\PY@bf=\relax%
    \let\PY@ul=\relax \let\PY@tc=\relax%
    \let\PY@bc=\relax \let\PY@ff=\relax}
\def\PY@tok#1{\csname PY@tok@#1\endcsname}
\def\PY@toks#1+{\ifx\relax#1\empty\else%
    \PY@tok{#1}\expandafter\PY@toks\fi}
\def\PY@do#1{\PY@bc{\PY@tc{\PY@ul{%
    \PY@it{\PY@bf{\PY@ff{#1}}}}}}}
\def\PY#1#2{\PY@reset\PY@toks#1+\relax+\PY@do{#2}}

\expandafter\def\csname PY@tok@gd\endcsname{\def\PY@tc##1{\textcolor[rgb]{0.63,0.00,0.00}{##1}}}
\expandafter\def\csname PY@tok@gu\endcsname{\let\PY@bf=\textbf\def\PY@tc##1{\textcolor[rgb]{0.50,0.00,0.50}{##1}}}
\expandafter\def\csname PY@tok@gt\endcsname{\def\PY@tc##1{\textcolor[rgb]{0.00,0.27,0.87}{##1}}}
\expandafter\def\csname PY@tok@gs\endcsname{\let\PY@bf=\textbf}
\expandafter\def\csname PY@tok@gr\endcsname{\def\PY@tc##1{\textcolor[rgb]{1.00,0.00,0.00}{##1}}}
\expandafter\def\csname PY@tok@cm\endcsname{\let\PY@it=\textit\def\PY@tc##1{\textcolor[rgb]{0.25,0.50,0.50}{##1}}}
\expandafter\def\csname PY@tok@vg\endcsname{\def\PY@tc##1{\textcolor[rgb]{0.10,0.09,0.49}{##1}}}
\expandafter\def\csname PY@tok@m\endcsname{\def\PY@tc##1{\textcolor[rgb]{0.40,0.40,0.40}{##1}}}
\expandafter\def\csname PY@tok@mh\endcsname{\def\PY@tc##1{\textcolor[rgb]{0.40,0.40,0.40}{##1}}}
\expandafter\def\csname PY@tok@go\endcsname{\def\PY@tc##1{\textcolor[rgb]{0.53,0.53,0.53}{##1}}}
\expandafter\def\csname PY@tok@ge\endcsname{\let\PY@it=\textit}
\expandafter\def\csname PY@tok@vc\endcsname{\def\PY@tc##1{\textcolor[rgb]{0.10,0.09,0.49}{##1}}}
\expandafter\def\csname PY@tok@il\endcsname{\def\PY@tc##1{\textcolor[rgb]{0.40,0.40,0.40}{##1}}}
\expandafter\def\csname PY@tok@cs\endcsname{\let\PY@it=\textit\def\PY@tc##1{\textcolor[rgb]{0.25,0.50,0.50}{##1}}}
\expandafter\def\csname PY@tok@cp\endcsname{\def\PY@tc##1{\textcolor[rgb]{0.74,0.48,0.00}{##1}}}
\expandafter\def\csname PY@tok@gi\endcsname{\def\PY@tc##1{\textcolor[rgb]{0.00,0.63,0.00}{##1}}}
\expandafter\def\csname PY@tok@gh\endcsname{\let\PY@bf=\textbf\def\PY@tc##1{\textcolor[rgb]{0.00,0.00,0.50}{##1}}}
\expandafter\def\csname PY@tok@ni\endcsname{\let\PY@bf=\textbf\def\PY@tc##1{\textcolor[rgb]{0.60,0.60,0.60}{##1}}}
\expandafter\def\csname PY@tok@nl\endcsname{\def\PY@tc##1{\textcolor[rgb]{0.63,0.63,0.00}{##1}}}
\expandafter\def\csname PY@tok@nn\endcsname{\let\PY@bf=\textbf\def\PY@tc##1{\textcolor[rgb]{0.00,0.00,1.00}{##1}}}
\expandafter\def\csname PY@tok@no\endcsname{\def\PY@tc##1{\textcolor[rgb]{0.53,0.00,0.00}{##1}}}
\expandafter\def\csname PY@tok@na\endcsname{\def\PY@tc##1{\textcolor[rgb]{0.49,0.56,0.16}{##1}}}
\expandafter\def\csname PY@tok@nb\endcsname{\def\PY@tc##1{\textcolor[rgb]{0.00,0.50,0.00}{##1}}}
\expandafter\def\csname PY@tok@nc\endcsname{\let\PY@bf=\textbf\def\PY@tc##1{\textcolor[rgb]{0.00,0.00,1.00}{##1}}}
\expandafter\def\csname PY@tok@nd\endcsname{\def\PY@tc##1{\textcolor[rgb]{0.67,0.13,1.00}{##1}}}
\expandafter\def\csname PY@tok@ne\endcsname{\let\PY@bf=\textbf\def\PY@tc##1{\textcolor[rgb]{0.82,0.25,0.23}{##1}}}
\expandafter\def\csname PY@tok@nf\endcsname{\def\PY@tc##1{\textcolor[rgb]{0.00,0.00,1.00}{##1}}}
\expandafter\def\csname PY@tok@si\endcsname{\let\PY@bf=\textbf\def\PY@tc##1{\textcolor[rgb]{0.73,0.40,0.53}{##1}}}
\expandafter\def\csname PY@tok@s2\endcsname{\def\PY@tc##1{\textcolor[rgb]{0.73,0.13,0.13}{##1}}}
\expandafter\def\csname PY@tok@vi\endcsname{\def\PY@tc##1{\textcolor[rgb]{0.10,0.09,0.49}{##1}}}
\expandafter\def\csname PY@tok@nt\endcsname{\let\PY@bf=\textbf\def\PY@tc##1{\textcolor[rgb]{0.00,0.50,0.00}{##1}}}
\expandafter\def\csname PY@tok@nv\endcsname{\def\PY@tc##1{\textcolor[rgb]{0.10,0.09,0.49}{##1}}}
\expandafter\def\csname PY@tok@s1\endcsname{\def\PY@tc##1{\textcolor[rgb]{0.73,0.13,0.13}{##1}}}
\expandafter\def\csname PY@tok@kd\endcsname{\let\PY@bf=\textbf\def\PY@tc##1{\textcolor[rgb]{0.00,0.50,0.00}{##1}}}
\expandafter\def\csname PY@tok@sh\endcsname{\def\PY@tc##1{\textcolor[rgb]{0.73,0.13,0.13}{##1}}}
\expandafter\def\csname PY@tok@sc\endcsname{\def\PY@tc##1{\textcolor[rgb]{0.73,0.13,0.13}{##1}}}
\expandafter\def\csname PY@tok@sx\endcsname{\def\PY@tc##1{\textcolor[rgb]{0.00,0.50,0.00}{##1}}}
\expandafter\def\csname PY@tok@bp\endcsname{\def\PY@tc##1{\textcolor[rgb]{0.00,0.50,0.00}{##1}}}
\expandafter\def\csname PY@tok@c1\endcsname{\let\PY@it=\textit\def\PY@tc##1{\textcolor[rgb]{0.25,0.50,0.50}{##1}}}
\expandafter\def\csname PY@tok@kc\endcsname{\let\PY@bf=\textbf\def\PY@tc##1{\textcolor[rgb]{0.00,0.50,0.00}{##1}}}
\expandafter\def\csname PY@tok@c\endcsname{\let\PY@it=\textit\def\PY@tc##1{\textcolor[rgb]{0.25,0.50,0.50}{##1}}}
\expandafter\def\csname PY@tok@mf\endcsname{\def\PY@tc##1{\textcolor[rgb]{0.40,0.40,0.40}{##1}}}
\expandafter\def\csname PY@tok@err\endcsname{\def\PY@bc##1{\setlength{\fboxsep}{0pt}\fcolorbox[rgb]{1.00,0.00,0.00}{1,1,1}{\strut ##1}}}
\expandafter\def\csname PY@tok@mb\endcsname{\def\PY@tc##1{\textcolor[rgb]{0.40,0.40,0.40}{##1}}}
\expandafter\def\csname PY@tok@ss\endcsname{\def\PY@tc##1{\textcolor[rgb]{0.10,0.09,0.49}{##1}}}
\expandafter\def\csname PY@tok@sr\endcsname{\def\PY@tc##1{\textcolor[rgb]{0.73,0.40,0.53}{##1}}}
\expandafter\def\csname PY@tok@mo\endcsname{\def\PY@tc##1{\textcolor[rgb]{0.40,0.40,0.40}{##1}}}
\expandafter\def\csname PY@tok@kn\endcsname{\let\PY@bf=\textbf\def\PY@tc##1{\textcolor[rgb]{0.00,0.50,0.00}{##1}}}
\expandafter\def\csname PY@tok@mi\endcsname{\def\PY@tc##1{\textcolor[rgb]{0.40,0.40,0.40}{##1}}}
\expandafter\def\csname PY@tok@gp\endcsname{\let\PY@bf=\textbf\def\PY@tc##1{\textcolor[rgb]{0.00,0.00,0.50}{##1}}}
\expandafter\def\csname PY@tok@o\endcsname{\def\PY@tc##1{\textcolor[rgb]{0.40,0.40,0.40}{##1}}}
\expandafter\def\csname PY@tok@kr\endcsname{\let\PY@bf=\textbf\def\PY@tc##1{\textcolor[rgb]{0.00,0.50,0.00}{##1}}}
\expandafter\def\csname PY@tok@s\endcsname{\def\PY@tc##1{\textcolor[rgb]{0.73,0.13,0.13}{##1}}}
\expandafter\def\csname PY@tok@kp\endcsname{\def\PY@tc##1{\textcolor[rgb]{0.00,0.50,0.00}{##1}}}
\expandafter\def\csname PY@tok@w\endcsname{\def\PY@tc##1{\textcolor[rgb]{0.73,0.73,0.73}{##1}}}
\expandafter\def\csname PY@tok@kt\endcsname{\def\PY@tc##1{\textcolor[rgb]{0.69,0.00,0.25}{##1}}}
\expandafter\def\csname PY@tok@ow\endcsname{\let\PY@bf=\textbf\def\PY@tc##1{\textcolor[rgb]{0.67,0.13,1.00}{##1}}}
\expandafter\def\csname PY@tok@sb\endcsname{\def\PY@tc##1{\textcolor[rgb]{0.73,0.13,0.13}{##1}}}
\expandafter\def\csname PY@tok@k\endcsname{\let\PY@bf=\textbf\def\PY@tc##1{\textcolor[rgb]{0.00,0.50,0.00}{##1}}}
\expandafter\def\csname PY@tok@se\endcsname{\let\PY@bf=\textbf\def\PY@tc##1{\textcolor[rgb]{0.73,0.40,0.13}{##1}}}
\expandafter\def\csname PY@tok@sd\endcsname{\let\PY@it=\textit\def\PY@tc##1{\textcolor[rgb]{0.73,0.13,0.13}{##1}}}

\def\PYZbs{\char`\\}
\def\PYZus{\char`\_}
\def\PYZob{\char`\{}
\def\PYZcb{\char`\}}
\def\PYZca{\char`\^}
\def\PYZam{\char`\&}
\def\PYZlt{\char`\<}
\def\PYZgt{\char`\>}
\def\PYZsh{\char`\#}
\def\PYZpc{\char`\%}
\def\PYZdl{\char`\$}
\def\PYZhy{\char`\-}
\def\PYZsq{\char`\'}
\def\PYZdq{\char`\"}
\def\PYZti{\char`\~}
% for compatibility with earlier versions
\def\PYZat{@}
\def\PYZlb{[}
\def\PYZrb{]}
\makeatother


    % Exact colors from NB
    \definecolor{incolor}{rgb}{0.0, 0.0, 0.5}
    \definecolor{outcolor}{rgb}{0.545, 0.0, 0.0}



    
    % Prevent overflowing lines due to hard-to-break entities
    \sloppy 
    % Setup hyperref package
    \hypersetup{
      breaklinks=true,  % so long urls are correctly broken across lines
      colorlinks=true,
      urlcolor=blue,
      linkcolor=darkorange,
      citecolor=darkgreen,
      }
    % Slightly bigger margins than the latex defaults
    
    \geometry{verbose,tmargin=1in,bmargin=1in,lmargin=1in,rmargin=1in}
    
    

    \begin{document}
      \author{ Zeren Lin  }
    
    \maketitle
    
    

    
    Where is PS8?

    \section{BEC in d-dimensions}\label{bec-in-d-dimensions}

    \begin{enumerate}
\def\labelenumi{(\alph{enumi})}
\itemsep1pt\parskip0pt\parsep0pt
\item
  For ideal Bose gas


$\frac{N}{V} = \frac{1}{V}\frac{\zeta}{1-\zeta}+ \int \frac{d^dk}{(2 \pi)^d} \frac{1}{e^{\epsilon_k \beta}/\zeta-1}=\frac{1}{V}\frac{\zeta}{1-\zeta}+  \int \frac{d^dk}{(2 \pi)^d}\zeta e^{-\epsilon_k \beta}(1+\zeta e^{-\epsilon_k \beta}+ (\zeta e^{-\epsilon_k \beta})^2 + ...) =\frac{1}{V} \frac{\zeta}{1-\zeta}+\sum_{n=1}^{\infty} \int \frac{d^dk}{(2 \pi)^d} e^{-n\epsilon_k \beta}\zeta^n$

Define $ I_1 = \int \frac{d^dk}{(2 \pi)^d} e^{-\epsilon_k \beta} = \int \frac{d^dk}{(2 \pi)^d} e^{-\frac{1}{2 \alpha} k^4 \beta}$, then

$I_n = \int \frac{d^dk}{(2 \pi)^d} e^{-n\epsilon_k \beta} = \int \frac{d^dk}{(2 \pi)^d} e^{-\frac{n}{2 \alpha} k^4 \beta} =\frac{1}{n^{d/4}}\int \frac{d^d(kn^{1/4})}{(2 \pi)^d} e^{-\frac{1}{2 \alpha} {(k n^{1/4})}^4 \beta} $

Therefore

$\frac{N}{V} =\frac{1}{V}\frac{\zeta}{1-\zeta}+ I_1\sum_{n=1}^{\infty}\frac{\zeta^n}{n^{d/4}}$

Note that
$I_1 \le \int_{k \le 1} \frac{d^dk}{(2 \pi)^d} e^{-\frac{1}{2 \alpha} k^4 \beta}+\int_{k>1} \frac{d^dk}{(2 \pi)^d} e^{-\frac{1}{2 \alpha} k^2 \beta} \le \mbox{const} + \int \frac{d^dk}{(2 \pi)^d} e^{-\frac{1}{2 \alpha} k^2 \beta} = \mbox{const} + \frac{1}{\lambda_T^d} < \infty$

When $d<5$, the sum will diverge if $\zeta \rightarrow1$, which means
$\zeta$ can't reach $1$. Therefore, $d=5$ is the minimum dimension in
which BEC can occur
\end{enumerate}

    \begin{enumerate}
\def\labelenumi{(\alph{enumi})}
\setcounter{enumi}{1}
\itemsep1pt\parskip0pt\parsep0pt
\item
  If BEC can occur, at $T=T_{BEC}$


$\frac{N}{V} = I_1\sum_{n=1}^{\infty}\frac{1}{n^{d/4}}$

By again looking at $I_1$

$ I_1  = \int \frac{d^dk}{(2 \pi)^d} e^{-\frac{1}{2 \alpha} k^4 \beta} = \frac{1}{\beta^{d/4}}\int \frac{d^d(k\beta^{1/4})}{(2 \pi)^d} e^{-\frac{1}{2 \alpha} {(k \beta^{1/4})}^4 }=\frac{a}{\beta^{d/4}}$, where $a$ is a proportionality constant.

Then

$\frac{N}{V}= b T_{BEC}^{d/4}$

$T_{BEC} \propto (\frac{N}{V})^{4/d}$

when $d=5$, $T_{BEC} \propto (\frac{N}{V})^{4/5}$
\end{enumerate}

    \begin{enumerate}
\def\labelenumi{(\alph{enumi})}
\setcounter{enumi}{2}
\itemsep1pt\parskip0pt\parsep0pt
\item
  If the dispersion relation changes to
  $\epsilon_k = \frac{1}{2 \alpha} |k^2-k_0^2|^2$


$\frac{N}{V} =\frac{1}{V} \frac{\zeta}{1-\zeta}+\sum_{n=1}^{\infty} \int \frac{d^dk}{(2 \pi)^d} e^{-n\epsilon_k \beta}\zeta^n$

where $I_1^{'}$ is now

 $ I_1^{'} = \int \frac{d^dk}{(2 \pi)^d} e^{-\epsilon_k \beta} = \int \frac{d^dk}{(2 \pi)^d} e^{-\frac{1}{2 \alpha} |k^2-k_0^2|^2 \beta} $
 

$I_n ^{'}= \int \frac{d^dk}{(2 \pi)^d} e^{-n\epsilon_k \beta} = \int \frac{d^dk}{(2 \pi)^d} e^{-\frac{n}{2 \alpha} |k^2-k_0^2|^2 \beta}  $

where

$\int \frac{d^dk}{(2 \pi)^d} e^{-\frac{n}{2 \alpha} |k^2-k_0^2|^2 \beta} \ge \int_{k^2>k_0^2} \frac{d^dk}{(2 \pi)^d} e^{-\frac{n}{2 \alpha} (k^4+k_0^4-2k_0^2k^2) \beta} \ge \int_{k^2>k_0^2} \frac{d^dk}{(2 \pi)^d} e^{-\frac{n}{2 \alpha} (2k^4-2k_0^4) \beta} = e^{\frac{n}{ \alpha}k_0^4 \beta}\int_{k^2>k_0^2} \frac{d^dk}{(2 \pi)^d} e^{-\frac{n}{ \alpha} k^4 \beta} = \frac{1}{n^{d/4}}e^{\frac{n}{ \alpha}k_0^4 \beta}\int_{(k(n)^{1/4})^2>(k_0(n)^{1/4})^2} \frac{d^d(k(n)^{1/4})}{(2 \pi)^d} e^{-\frac{1}{ \alpha} (k(n)^{1/4})^4 \beta} \ge \frac{1}{n^{d/4}}e^{\frac{n}{ \alpha}k_0^4 \beta}\int_{(k(n)^{1/4})^2>k_0^2} \frac{d^d(k(n)^{1/4})}{(2 \pi)^d} e^{-\frac{1}{ \alpha} (k(n)^{1/4})^4 \beta} = C \frac{1}{n^{d/4}}e^{\frac{n}{ \alpha}k_0^4 \beta}$

where

$C= \int_{k^2>k_0^2} \frac{d^dk}{(2 \pi)^d} e^{-\frac{1}{ \alpha} k^4 \beta} < \infty$

Therefore

$\frac{N}{V} \ge \frac{1}{V} \frac{\zeta}{1-\zeta}+C \sum_{n=1}^{\infty} \frac{\zeta^n}{n^{d/4}}e^{\frac{n}{ \alpha}k_0^4 \beta}$

The sum will always diverge as $\zeta \rightarrow1$, BEC can never occur
\end{enumerate}

    \section{Virial expansion for interacting
gases}\label{virial-expansion-for-interacting-gases}

    \begin{enumerate}
\def\labelenumi{(\alph{enumi})}
\itemsep1pt\parskip0pt\parsep0pt
\item
  Grand canonical partition function


$\mathcal{Z} = \sum_N \sum_{i} e^{-\beta(E_{i}-\mu N)} = \sum_N \zeta^N  e^{-\beta \{ \sum_{i\in \{1,...N\}} \frac{P_{i}^2}{2m} + \sum_{i<j\in \{1,...N\} }V(r_{ij})\} } = \sum_N \zeta^N  \Pi_{i\in \{1,...N\}} e^{-\beta  \frac{P_{i}^2}{2m}} \Pi_{i<j\in \{1,...N \}}e^{ -\beta V(r_{ij})}$

where

$\Pi_{i\in \{1,...N\}} e^{-\beta  \frac{P_{i}^2}{2m}}  = (\frac{1}{\lambda_T^3})^N / N!$

$\Pi_{i<j\in \{1,...N\}}e^{ -\beta V(r_{ij})}= \Pi_{i<j\in \{1,...N\}}(1 + e^{ -\beta V(r_{ij})} -1) \simeq \sum_{i<j\in \{1,...N\}}(1 + e^{ -\beta V(r_{ij})} -1) = V^N + V^{N-2}  \frac{N(N-1)}{2}\int d^3r_1 \int d^3r_2 (e^{ -\beta V(r_{12})} -1)= V^N + V^{N-1}  \frac{N(N-1)}{2}\int d^3r_{12}  (e^{ -\beta V(r_{12})} -1) $

Thus

$\mathcal{Z} = \sum_N \zeta^N  (\frac{1}{\lambda_T^3})^N / N!\{V^N + V^{N-1}  \frac{N(N-1)}{2}\int d^3r_{12}  (e^{ -\beta V(r_{12})} -1)\} = \sum_N \{ \zeta^N  (\frac{V}{\lambda_T^3})^N / N!+ \zeta^{N-2}  (\frac{V}{\lambda_T^3})^{N-2} / (N-2)! (\zeta \frac{1}{\lambda_T^3})^2 \frac{V}{2}\int d^3r_{12}  (e^{ -\beta V(r_{12})} -1)\} = e^{\zeta \frac{V}{\lambda_T^3}}(1+ (\zeta  \frac{1}{\lambda_T^3})^2 \frac{V}{2 }\int d^3r_{12}  (e^{ -\beta V(r_{12})} -1) )$

Define $\frac{1}{\lambda_T^3}=Z_1$

$ \mathcal{Z}\simeq e^{\zeta Z_1 V}\{1+ \zeta^2  Z_1^2 \frac{V}{2 }  \int d^3r_{12}  (e^{ -\beta V(r_{12})} -1)\}$

Or

$ \mathcal{Z}\simeq e^{\zeta Z_1 V}\{1+ \zeta^2  Z_1^2 \frac{1}{2 }  \int d^3r_1 \int d^3r_2  (e^{ -\beta V(r_{12})} -1)\}$
\end{enumerate}

    \begin{enumerate}
\def\labelenumi{(\alph{enumi})}
\setcounter{enumi}{1}
\itemsep1pt\parskip0pt\parsep0pt
\item
  Grand canonical potential


$w= -T\mbox{ln}\mathcal{Z} \simeq -T \zeta Z_1 V-T\mbox{ln}\{1+ \zeta^2  Z_1^2 \frac{V}{2} \int d^3r_{12}  (e^{ -\beta V(r_{12})} -1)\}=-PV$

$N = -(\frac{\partial w}{\partial \mu})_{T, V} = -\beta \zeta (\frac{\partial w}{\partial \zeta})_{T, V}\simeq \zeta   Z_1 V+ \frac{\zeta^2  Z_1^2 V\int d^3r_{12}  (e^{ -\beta V(r_{12})} -1)}{1+ \zeta^2  Z_1^2 \frac{V}{2}\int d^3r_{12}  (e^{ -\beta V(r_{12})} -1)} \simeq  \zeta   Z_1 V (1+  \zeta Z_1\frac{1}{2} \int d^3r_{12}  (e^{ -\beta V(r_{12})} -1))$

Therefore

$P \simeq T \zeta Z_1 \{1+ \zeta Z_1\frac{1}{2}\int d^3r_{12}  (e^{ -\beta V(r_{12})} -1)\}$

where

$ \zeta   Z_1 \simeq \frac{N}{V} \{ 1-  \frac{N}{V}\int d^3r_{12}  (e^{ -\beta V(r_{12})} -1)\}$

Then

$P  \simeq T \frac{N}{ V} \{1-\frac{N}{ V} \frac{1}{2}\int d^3r_{12}  (e^{ -\beta V(r_{12})} -1)\}$

For the given interaction

$\frac{1}{2}  \int d^3r_{12} (e^{ -\beta V(r_{12})} -1) =-2\pi   \int_0^b r^2 dr+ 2\pi  \int_b^{\infty} dr r^2(e^{ \beta \frac{a}{r^6}} -1) \sim  -2\pi  \frac{1}{3} b^3 + 2\pi  \int_b^{\infty} dr  \beta \frac{a}{r^4} = 2\pi  (\beta \frac{a}{3b^3} -\frac{1}{3} b^3)  $

Therefore

$P \simeq T \frac{N}{ V} \{1-\frac{N}{ V} 2\pi  (\beta \frac{a}{3b^3} -\frac{1}{3} b^3)\} =  T \{\frac{N}{V}-2 \pi \frac{N^2}{V^2} (\beta \frac{a}{3b^3} -\frac{1}{3} b^3) \}$

$P+ 2 \pi \frac{N^2}{V^2}\frac{a}{3b^3} = T \frac{N}{V}(1+\frac{2 \pi N}{3 V} b^3)\simeq \frac{N T}{V-\frac{2 \pi N}{3} b^3} $

And

$B_1 = -2\pi  (\beta \frac{a}{3b^3} -\frac{1}{3} b^3)$
\end{enumerate}


    % Add a bibliography block to the postdoc
    
    
    
    \end{document}
